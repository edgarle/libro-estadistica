% Preámbulo modificado de \cite{Borbon2013} y formato de PLOS version 3.1 (Febrero 2015)


\documentclass{article}
%\usepackage[top=0.85in,left=2.75in,footskip=0.75in]{geometry}

% Traduce los encabezados de las plantillas y permite la separación de palabras
\usepackage[spanish]{babel}

% Soporte para acentos en castellano
\usepackage[utf8]{inputenc}
\usepackage[T1]{fontenc}

% paquetes amsmath y amssymb útiles para fórmulas matemáticas y símbolosus
%\usepackage{amsmath,amssymb}

% paquete cite, para citas en el texto principal
%\usepackage{cite}
%
%% usar nameref para enlaces dentro del textom, p. ej. ver Información adicional para más info
%\usepackage{nameref,hyperref}
%
% line numbers
\usepackage[right]{lineno}
%
%% desabilita que dos letras se junten (f.)
%\usepackage{microtype}
%\DisableLigatures[f]{encoding = *, family = * }
%
%% paquete rotating para cuadros horizontales
%\usepackage{rotating}
%
%% incluir el macro \verb en \caption de las figuras 
%% http://tex.stackexchange.com/a/8814
%% p. ej. \cprotect\caption{blah \verb|#$%^&| blah...}
%\usepackage{cprotect}
%
%% remover los comentarios para espaciado doble
%%\usepackage{setspace}
%%\doublespacing
%
%% Text layout
%\raggedright
%\setlength{\parindent}{0.5cm}
%\textwidth 5.25in
%\textheight 8.75in
%
%% Negrita para 'Figura #' en la leyenda y la separa del título/leyenda con un punto
%% Las leyendas estarán justificadas al centro
%\usepackage[aboveskip=1pt,labelfont=bf,labelsep=period,justification=raggedright,singlelinecheck=off]{caption}
%
%% remover si se quiere usar el estilo de PLoS para BiBTeX
%%\bibliographystyle{plos2015}
%
%% Remove brackets from numbering in List of References
%%\makeatletter
%%\renewcommand{\@biblabel}[1]{\quad#1.}
%%\makeatother
%
%% para la fecha de compilación
%\date{}
%
%% remover y editar para encabezado y pie de página con logo
%%\usepackage{lastpage,fancyhdr,graphicx}
%%\usepackage{epstopdf}
%%\pagestyle{myheadings}
%%\pagestyle{fancy}
%%\fancyhf{}
%%\lhead{\includegraphics[width=2.0in]{PLOS-submission.eps}}
%%\rfoot{\thepage/\pageref{LastPage}}
%%\renewcommand{\footrule}{\hrule height 2pt \vspace{2mm}}
%%\fancyheadoffset[L]{2.25in}
%%\fancyfootoffset[L]{2.25in}
%%\lfoot{\sf PLOS}
%
%%% Include all macros below
%
%\newcommand{\lorem}{{\bf LOREM}}
%\newcommand{\ipsum}{{\bf IPSUM}}
%
%%% END MACROS SECTION
%
%
%% para que beamer (si se usa)no cambie las fuentes
%%\usefonttheme{professionalfonts}

\begin{document}
\vspace*{0.35in}

\begin{flushleft}
{\Large
\textbf\newline{Estadística para biología con R}
}
\newline
% Insert author names, affiliations and corresponding author email (do not include titles, positions, or degrees).
\\
Edgar E. Gareca\textsuperscript{1,*},
David G. Rossiter\textsuperscript{2},
Otros son bienvenidos{3}
\\
\bigskip
\bf{1} Centro de Biodiversidad y Genética, Universidad Mayor de San Simón, Cochabamba, Estado Plurinacional de Bolivia
\\
\bf{2} Soil \& Crop Sciences Section, Cornell University, Ithaca NY, United States of America
\\
\bf{3} Colaboradores invitados
\\
\bigskip
* edgargareca@gmail.com

\end{flushleft}

\linenumbers

\section*{Contenidos}
	\tableofcontents

\section{Preparando el entorno de R}
Siga los siguientes pasos:
\begin{enumerate}
\item Instalar el programa de base R (sección 1.1)
\item Instalar el IDE RStudio (sección 2)
\item Instalar parquetes adicionales de R (sección 2.1)
\item Configurar un proyecto en RStudio (sección 2.2)
\end{enumerate}
El llevar a cabo estos pasos dependerá de su sistema operativo. En este caso solo consideraremos a Microsoft Windows. Si utiliza otros sistemas contacte a los autores.

\subsection{Instalando el programa base de R}
La primera actividad que hará para utilizar R, es necesariamente instalarlo. Si usted tiene una versión anterior de R puede instalar una nueva versión en su ordenador. La nueva versión se instalará en una carpeta nueva y tendrá su propia carpeta de paquetes instalados. En cualquiera de los dos casos (nueva instalación o nueva versión de R) debe seguir los siguientes pasos:
\begin{enumerate}
\item Conecte su ordenador a la red internacional (internet)
\item Siga el enlace que le dirigirá a la versión actual del instalador de R para Windows y automáticamente comenzará a descargar el software en una ubicación temporal de su sistema: http://mirrors.dotsrc.org/cran/bin/windows/base/release.htm El programa para instalar R en Microsoft Windows se llamará R-3.3.3-win.exe\footnote{la versión al 11 de marzo de 2017}; el nombre cambia con la versión.

Si usted cuenta con otro sistema operativo visite CRAN (las siglas en inglés de ``la red de archivos completa de R'')\footnote{http://cran.r-project.org}

\item Ejecute el instalador; éste instalará R en su máquina.

\textbf{Nota}: necesitará \textit{privilegios de administrador} para instalar R u otro programa.
\item Un ícono para R se creará en su Escritorio al igual que un enlace en su menú ``Inicio''.
\end{enumerate}

\begin{itemize}
\item Confirme que quiere instalar el programa descargado de un sitio inseguro.
\item Seleccione el idioma durante la instalación
\item Inicie la instalación
\item Siguiente en Información acerca de la licencia
\item Siguiente en Seleccione la carpeta de destino (se puede cambiar si hace clic en ``Examinar\ldots''
\item Siguiente en Seleccione los componentes
\item Seleccione ``No'' en Opciones de configuración
\item Siguiente en Seleccione la Carpeta del menú inicio
\item Seleccione las Tareas adicionales que le interese y presione Siguiente
\item Espere mientras se instala R
\item Presione en Finalizar
\end{itemize}

\section{Instalando RStudio}
\section{Ejemplo de uso de RStudio y R}
\section{Accediendo a los códigos de R para los ejercicios}

\end{document}